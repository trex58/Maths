\documentclass[latin1]{quadrature}

%\usepackage[utf8]{inputenc}
\usepackage[T1]{fontenc}
\usepackage{babel}

% Pour fabriquer la bibliographie sous MixTex :
%   Fabriquer avec pdfLatex, puis BibTex, pdfLatex, et encore pdfLatex.

\newcommand{\PMod}[1]{\!\!\pmod{#1}}

\newcommand{\Legendre}[2]{(\leavevmode\kern.1em \raise.5ex\hbox{\the\scriptfont0 #1}\kern-.15em
/\kern-.15em\lower.45ex\hbox{\the\scriptfont0 #2})}
\newcommand{\LLegendre}[2]{\big(\leavevmode\kern.05em \raise.5ex\hbox{#1}\kern-.2em
/\kern-.15em\lower.55ex\hbox{{#2}}\big)}
\newcommand{\Frac}[2]{\leavevmode\kern.1em \raise.5ex\hbox{\the\scriptfont0 #1}\kern-.15em
/\kern-.15em\lower.45ex\hbox{\the\scriptfont0 #2}}
%\newcommand{\FFrac}[2]{\leavevmode\kern.05em \raise.5ex\hbox{#1}\kern-.2em
%/\kern-.15em\lower.55ex\hbox{#2}}

\newcommand{\mylegendre}[2]{\genfrac{(}{)}{}{}{#1}{#2}}


\newtheorem{theorem}{Theorem}
\newtheorem{conjecture}{Conjecture}


\newif\ifenfrancais


%	 À commenter pour passer en anglais
\enfrancaistrue
\ifenfrancais
\else
\fi


\numero{42}
\magazinemonth{avril}
\magazineyear{2024}

\begin{document}
\ifenfrancais
\begin{article}[%
subarticle=Th\'eorie des Nombres,
title=Les nombres de Wagstaff (v3),
subtitle={Tout ce que vous avez toujours voulu savoir sur eux (sans oser demander)},
author=Tony \textsc{Reix}\thanks{tony.reix@laposte.net},
abstract={ Apr\`es une pr\'esentation des nombres de Wagstaff, l'article rappelle les m\'ethode connues permettant d'en trouver des PRPs (PRobably Prime) et indique le record actuel. Puis sont d\'ecrites les caract\'eristiques des DiGraphs (Directed Graph) g\'en\'er\'es sous $x^2-2$ pour les nombres de Mersenne, Fermat, et Wagstaff. On donne le lien surprenant entre le nombre de Cycles de longueur $q-1$ et $q-2$ du DiGraph sous $x^2-2$ modulo un nombre de Wagstaff premier et la s\' erie A165921 d'OEIS d\'efinie \`a partir des polyn\^omes irr\'eductibles. Puis on montre que le test de primalit\'e des nombres de Fermat construit \`a partir de la th\'eorie des Courbes Elliptiques semble fonctionner \'egalement pour les nombres de Wagstaff, avec de nouveau un lien surprenant entre le nombre de Cycles de longueur $q-2$ du DiGraph sous $dst(x)=\frac{\displaystyle x^4+2x^2+1}{\displaystyle 4(x^3-x)}$ modulo un nombre de Wagstaff premier et la s\' erie A165921 d'OEIS. Enfin, on fournit la visualisation de plusieurs DiGraph sous $dst(x)$ modulo des nombres de Fermat et de Wagstaff.

Le but de cet article est de susciter l'int\'er\^et pour les nombres de Wagstaff et pour la recherche d'une m\'ethode rapide de preuve de primalit\'e pour ces nombres.}
]
\else
\begin{article}[%
subarticle=Number Theory,
title=The Wagstaff numbers (v3),
subtitle={Everything You Always Wanted to Know About Them (But Were Afraid to Ask)},
author=Tony \textsc{Reix}\thanks{tony.reix@laposte.net},
abstract={After a description of the Wagstaff numbers, the paper recalls the known methods used for finding Wagstaff PRPs (PRobably Prime) and it provides the current world record. Then it describes the characteristics of the DiGraphs (Directed Graph) generated by $x^2-2$ for the Mersenne, Fermat and Wagstaff numbers. It provides the weird link between the number of Cycles of length $q-1$ and $q-2$ of the DiGraph under $x^2-2$ modulo a Wagstaff prime and the OEIS series A165921 defined from irreducible polynomials. Then, it shows that a Primality Test for Fermat numbers built by means of the Elliptic Curve theory seems to work also for the Wagstaff numbers, and - again - it shows a surprising link between the number of Cycles of length $q-2$ of the DiGraph under $\frac{\displaystyle x^4+2x^2+1}{\displaystyle 4(x^3-x)}$ modulo a Wagstaff prime and  the OEIS series A165921. Last, it provides the graphic display of several DiGraphs under $\frac{\displaystyle x^4+2x^2+1}{\displaystyle 4(x^3-x)}$ modulo Fermat and Wagstaff primes, with details, showing the main BigTrees and main Cycles.
\newline

The goal of this paper is to create interest for the Wagstaff numbers and for the search for an efficient method for proving that a Wagstaff PRP is a true prime.}
]
\fi


\section{ Introduction }

\ifenfrancais
Les nombres de Wagstaff sont des cousins des nombres de Fermat et des cousins germains des nombres de Mersenne. Mais ils ne sont pas smooth comme eux... et on ne conna\^it donc pour le moment aucune m\'ethode permettant de prouver \textbf{rapidement} qu'un nombre de Wagstaff est premier. On sait uniquement trouver rapidement des PRP de Wagstaff au moyen de techniques souvent inspir\'ees du Lucas-Lehmer Test. On d\'ecrit ici une nouvelle m\'ethode potentielle, inspir\'ee d'un test r\'ecent de primalit\'e pour les nombres de Fermat bas\'e sur les Courbes Elliptiques.
\else
The Wagstaff numbers are cousins of the Fermat numbers and german cousins of the Mersenne numbers. However, they are not smooth as the others are... and thus no fast Primality Test is known for them. We only know fast PRP tests for them, often based on the Lucas-Lehmer Test technic. Here after is described a new potential method, based on a recent Primality Test for Fermat numbers built on Elliptic Curves.
\fi

%\vspace{-.2in}


\ifenfrancais
\section{ Les Nombres de Wagstaff}
\else
\section{ Wagstaff numbers }
\fi

\ifenfrancais
\subsection{ D\'efinition et statut }
\else
\subsection{ Definition and status }
\fi

\ifenfrancais
Un nombre de Wagstaff est d\'efini par :
\else
A Wagstaff number is:
\fi
 $W_q = \frac{\displaystyle 2^q+1}{\displaystyle3}$.
 
 \vspace{.05in}
 \ifenfrancais
 
$W_q$ premier implique : $q$ premier.
\else
If $W_q$ is prime, thus $q$ is prime.
\fi

\vspace{.03in}

\ifenfrancais
C'est aussi un 
\else
It is a
\fi
 RepUnit $R_n^{(b)} = \frac{\displaystyle b^n-1}{\displaystyle b-1}$
 \ifenfrancais
 avec 
\else
  with base
\fi
 $b=-2$ .

\vspace{.03in}

%	 q=29;W=(2^q+1)/3;w=0;for(i=0,(q-3)/2,w+=4^i);a=factor(w);w=2*w+1;b=factor(w);print(W," ",w);print(a);print(b)
%	178956971 178956971
%	[5, 1; 29, 1; 43, 1; 113, 1; 127, 1]
%	[59, 1; 3033169, 1]

\ifenfrancais
Nous avons: $W_q = 1+2 \displaystyle \sum_{i=0}^{\frac{q-3}{2}}4^i = 1 + 2qk $ .

%	And, with $q'=\frac{q-1}{2}$ and $M_k=2^k-1$ : $W_q = 1+2 \times M_{q'} \times W_{q'}$.

Et, avec $M_k=2^k-1$ : $W_q = 1+2 \displaystyle  M_{\frac{q-1}{2}} W_{\frac{q-1}{2}}$.

Ainsi, si $q$ est premier, il divise soit $M_{\frac{q-1}{2}}$ soit $W_{\frac{q-1}{2}}$.
\else
We have: $W_q = 1+2 \displaystyle \sum_{i=0}^{\frac{q-3}{2}}4^i = 1 + 2qk $ .

%	And, with $q'=\frac{q-1}{2}$ and $M_k=2^k-1$ : $W_q = 1+2 \times M_{q'} \times W_{q'}$.

And, with $M_k=2^k-1$ : $W_q = 1+2 \displaystyle  M_{\frac{q-1}{2}} W_{\frac{q-1}{2}}$. Thus $q$ prime divides either $M_{\frac{q-1}{2}}$ or $W_{\frac{q-1}{2}}$.
\fi

\vspace{.1in}

\ifenfrancais
Donc, un nombre de Wagstaff n'est pas \emph{smooth} (il ne s'\'ecrit pas sous la forme : $W_q=N\pm1$ o\`u $N$ est enti\`erement ou partiellement factoris\'e).
\else
Thus, a Wagstaff number is not \emph{smooth} (it does not write as: $W_q=N+1 \text{ or } N-1$ where $N$ is completely or partially factorized).
\fi


\vspace{.1in}

\ifenfrancais
Il existe 34 nombres de Wagstaff premiers connus et 10 PRPs (Probably Prime) de Wagstaff.
Les premi\`eres valeurs de $q$ donnant un nombre de Wagstaff premier sont : $3, 5, 7, 11, 13, 17, 19, 23, 31, \dots$ Le premier Wagstaff non-premier appara\^t avec $q=29$.
Voir la s\'equence A000978 du projet OEIS.
\else
There are 34 known Wagstaff primes and 10 Wagstaff PRPs (PRobably Prime).
Values of $q$ for first Wagstaff primes are: $3, 5, 7, 11, 13, 17, 19, 23, 31, \dots$  First non-prime Wagstaff number appears with $q=29$.
See sequence A000978 in the OEIS project.
\fi

\vspace{.1in}

\ifenfrancais
Actuellement, il est possible de prouver qu'un nombre de Wagstaff est premier en utilisant la m\'ethode ECPP distribu\'ee sur plusieurs ordinateurs. Par exemple, c'est ainsi que le nombre de Wagstaff $W_{42,737}$ (12,865 digits) a \'et\'e prouv\'e premier par Fran\c{c}ois Morain en 2007, apr\`es que je lui ai signal\'e que ce $W_q$ \'etait \`a port\'ee des outils informatiques de l'\'epoque. Le plus r\'ecent nombre de Wagstaff premier connu est $W_{127,031}$, d\'ecouvert en janvier 2023.
\else
For now, a Wagstaff PRP can be proved prime by using a distributed ECPP implementation, as it was done for $q = 42,737$ (12,865 digits) by Fran\c{c}ois Morain in 2007 after I warned him about this $W_q$ being withing range,  or for $q=127,031$ in January 2023.
\fi

\vspace{.1in}


\ifenfrancais
En 2010, en tant que membre du projet DUR (Vincent Diepeveen, Paul Underwood, Tony Reix), j'ai d\'ecouvert le candidat record de PRP de Wagstaff :  $W_{4.031.399}$, en utilisant le test Vrba-Reix impl\'ement\'e dans l'outil \emph{LLR} par Jean Penn\'e du projet GIMPS.
Puis, au moyen de tests compl\'ementaires, il fut prouv\'e comme \'etant un vrai PRP.
\else
In 2010, as part of a team (DUR project: Vincent Diepeveen, Paul Underwood, Tony Reix), I discovered the candidate Wagstaff PRP Record $W_{4,031,399}$,
using the Vrba-Reix test implemented in the \emph{LLR} tool by Jean Penn\'e of the GIMPS project.
Then, thanks to complementary tests, it was proved to be a full PRP.
\fi

\vspace{.1in}

\ifenfrancais
Ensuite, 3 nouveaux PRPs de Wagstaff ont \'et\'e d\'ecouverts, en partie en utilisant les r\'esultats du projet DUR sur la v\'erification des candidats: $W_{13.347.311}$, $W_{13.372.531}$ et $W_{15.135.397}$.
\else
Then, three new Wagstaff PRPs have been found: $W_{13,347,311}$, $W_{13,372,531}$ and $W_{15,135,397}$, partly using the work of our DUR project on checking Wagstaff prime exponents.
\fi


\ifenfrancais
\subsection{ Comment trouver un Wagstaff PRP }
\else
\subsection{ Search of Wagstaff PRPs }
\fi

\ifenfrancais
Depuis plus de 20 ans, plusieurs tests ont \'et\'e trouv\'es et prouv\'es (ou conjectur\'es) afin de d\'ecouvrir de nouveaux PRPs de Wagstaff, en utilisant plusieurs techniques (comme les s\'equences de Lucas ou Lehmer, utilis\'ees pour le test LLT pour les nombres de Mersenne ou pour un test semblable au LLT pour les nombres de Fermat).
Voici les principaux tests :
\else
For more than 20 years, several PRP tests for Wagstaff numbers were found and proven for finding Wagstaff PRPs (plus many conjectures), based on several technics (like the Lucas-Lehmer Sequences, used for the LLT test for Mersenne numbers and used for a LLT-similar test for Fermat numbers).
Here are some of them:
\fi

\vspace{.1in}

%	\ifenfrancais
%	Voici les principaux tests :
%	\else
%	Here are some of them:
%	\fi


\ifenfrancais
\begin{theorem}[Lifchitz Renaud \& Henri - Juillet 2000]
\ 
\newline
Soit $N_p=2^p+1$ et $W_p = \frac{N_p}{3}$ . Si $W_p$ est premier, alors on a : $25^{2^{p-1}} \equiv 25 \PMod{N_p}$ .
\end{theorem}
\else
\begin{theorem}[Lifchitz Renaud \& Henri - 2000 July]
\ 
\newline
$N_p=2^p+1$ and $W_p = \frac{N_p}{3}$ . If $W_p$ is a prime, then $25^{2^{p-1}} \equiv 25 \PMod{N_p}$ .
\end{theorem}
\fi

\vspace{-.1in}

\ifenfrancais
\begin{theorem}[Vrba Anton \& Reix Tony - \`ala LLT - Group Theory - Cycle du DiGraph]
\ 
\newline
Soit $S_{n+1} = S_{n}^2 - 2$ et $p$ premier $\geqslant 3$.
Si $W_p = \frac{2^p+1}{3}$ est premier, alors $S_p \equiv S_2 \PMod{W_p}$
avec $S_0=6$ .
\end{theorem}
\else
\begin{theorem}[Vrba Anton \& Reix Tony - \`ala LLT - Group Theory]
Let $S_{n+1} = S_{n}^2 - 2$ and $p$ be a prime larger than $3$.
If $W_p = \frac{2^p+1}{3}$ is a prime, then $S_p \equiv S_2 \PMod{W_p}$
where $S_0=6$ .
\end{theorem}
\fi

\vspace{-.1in}

\ifenfrancais
\begin{theorem}[Gerbicz Robert - \`ala LLT]
\ 
\newline
Soit $q \geqslant 5$ premier, et soit $p=W(q)= \frac{2^q+1}{3}$ un nombre premier (de Wagstaff),
alors, avec la s\'equence $S_0=\frac{3}{2}$ , $S_{k+1} = S_{k}^2 - 2$ , $S_q - S_1$ est divisible par $p$ .
\end{theorem}
\else
\begin{theorem}[Gerbicz Robert - \`ala LLT]
\ 
\newline
Let $q > 3$ a prime, and $p=W(q)= \frac{2^q+1}{3}$ is also prime (Wagstaff prime),
then for the sequence $S_0=\frac{3}{2}$ , $S_{k+1} = S_{k}^2 - 2$ , it is true that $S_q - S_1$ is divisible by $p$ .
\end{theorem}
\fi

\vspace{-.1in}

\ifenfrancais
\begin{theorem}[Reix - \`ala P\'epin - Preuve bas\'ee sur une S\'equence de Lucas]

Si $W_q = \frac{2^{\scriptstyle q}+1}{3}$ ($q \text{ premier} \geqslant 7$) est premier , alors on a :
 \ $7^{\frac{W_q-1}{2}} \equiv -1 \ \PMod{W_q}$ .
 \end{theorem}
\else
\begin{theorem}[Reix - \`ala P\'epin - Proof based on a Lucas Sequence]

If $W_q = \frac{2^{\scriptstyle q}+1}{3}$ ($q \text{ prime} \geqslant 7$) is a prime , then:
 \ $7^{\frac{W_q-1}{2}} \equiv -1 \ \PMod{W_q}$ .
\end{theorem}
\fi

\vspace{-.1in}

\ifenfrancais
\begin{theorem}[Reix - Preuve bas\'ee sur une S\'equence de Lucas - Cycle du DiGraph]
\ 
\newline
Si $W_q =\frac{2^{\scriptstyle q}+1}{3}$ ($q \text{ premier } \geqslant 7$) est premier, alors :
$S_{q} \equiv S_2 \ \PMod{W_q}$ ,
avec : \ $S_0 = 8$ et $S_n = (S_{n-1}-1)^2+1$ .
\end{theorem}
\else
\begin{theorem}[Reix - Proof based on a Lucas Sequence]
\ 
If $W_q =\frac{2^{\scriptstyle q}+1}{3}$ ($q \text{ prime} \geqslant 7$) is a prime, then:
$S_{q} \equiv S_2 \ \PMod{W_q}$ ,
with: \ $S_0 = 8$ and $S_n = (S_{n-1}-1)^2+1$ .
\end{theorem}
\fi

\vspace{-.1in}

\ifenfrancais
\begin{theorem}[Reix - \`ala LLT - Preuve bas\'ee sur une S\'equence de Lehmer - Cycle du DiGraph]
\ 
\newline
Si $W_q = \frac{ 2^{\scriptstyle q} + 1}{3}$ ($q$ premier $\geqslant 11$) est premier, alors :
$S_{\,q} \equiv S_{\,2}=1154 \ \PMod{W_q}$ ,
avec : \ $S_{\,0} = 6$ \ et \ $S_{\,i} = S_{\,i-1}^{\,\,2} - 2$ \ pour \ $i=1,2,3, ... \ q$ .
\end{theorem}
\else
\begin{theorem}[Reix - \`ala LLT - Proof based on a Lehmer Sequence]
\ 
If $W_q = \frac{ 2^{\scriptstyle q} + 1}{3}$ ($q$ prime $\geqslant 11$) is a prime, then:
$S_{\,q} \equiv S_{\,2}=1154 \ \PMod{W_q}$ ,
with: \ $S_{\,0} = 6$ \ and \ $S_{\,i} = S_{\,i-1}^{\,\,2} - 2$ \ for \ $i=1,2,3, ... \ q$ .
\end{theorem}
\fi

\vspace{-.1in}

\ifenfrancais
\begin{theorem}[Paul Underwood - \`ala LLT ]
\ 
\newline
Si $W_q = \frac{ 2^{\scriptstyle q} + 1}{3}$ ($q \equiv \pm 1 \PMod{6}$) est premier, alors :
$S_{\,q-2} \equiv \pm 4 \ \PMod{W_q}$ ,
avec : \ $S_{\,0} = 4$ \ et \ $S_{\,i} = S_{\,i-1}^{\,\,2} - 2$ \ pour \ $i=1,2,3, ... $ et $q$ premier $\geqslant 7$ .

%$4$ can be replaced by any number $a$ or $b$ generated by: $a=4;b=4;for(i=3,20,a=14*b-a;print(a);b=14*a-b;print(b))$)
\end{theorem}
\else
\begin{theorem}[Paul Underwood - \`ala LLT ]
\ 
\newline
If $W_q = \frac{ 2^{\scriptstyle q} + 1}{3}$ ($q \equiv \pm 1 \PMod{6}$) is a prime, then:
$S_{\,q-2} \equiv \pm 4 \ \PMod{W_q}$ ,
with: \ $S_{\,0} = 4$ \ and \ $S_{\,i} = S_{\,i-1}^{\,\,2} - 2$ \ for \ $i=1,2,3, ... $ and $q$ prime $\geqslant 7$ .

%$4$ can be replaced by any number $a$ or $b$ generated by: $a=4;b=4;for(i=3,20,a=14*b-a;print(a);b=14*a-b;print(b))$)
\end{theorem}
\fi

% forprime(q=3,10001,W=(2^q+1)/3;s=Mod(4,W);for(i=1,q-2,s=s^2-2);if(lift(s)==4,print(q," P ", q%6)))
% forprime(q=3,10001,W=(2^q+1)/3;s=Mod(4,W);for(i=1,q-1,s=s^2-2);if(lift(s)==W-4,print(q," P ", q%6)))

\ifenfrancais
\begin{conjecture}[Reix - \`ala LLT - Cycle du DiGraph]
\ 
\newline
Si $W_q = \frac{ 2^{\scriptstyle q} + 1}{3}$ ($q$ premier $\geqslant 7$) est premier, alors :
$S_{\,q-1} \equiv S_{\,0} \ \PMod{W_q}$ ,
avec : \ $S_{\,0} = 7^2+1/7^2$ \ et \ $S_{\,i} = S_{\,i-1}^{\,\,2} - 2$ \ pour \ $i=1,2,3, ... \ q-1$ .
\end{conjecture}
\else
\begin{conjecture}[Reix - \`ala LLT ]
\ 
\newline
If $W_q = \frac{ 2^{\scriptstyle q} + 1}{3}$ ($q$ prime $\geqslant 7$) is a prime, then:
$S_{\,q-1} \equiv S_{\,0} \ \PMod{W_q}$ ,
with: \ $S_{\,0} = 7^2+1/7^2$ \ and \ $S_{\,i} = S_{\,i-1}^{\,\,2} - 2$ \ for \ $i=1,2,3, ... \ q-1$ .
\end{conjecture}
\fi

%	Ca ne semble pas marcher !!!!
%	q=11;W=(2^q+1)/3;END=Mod(5-9,W);x=Mod(4,W);print(0," ",lift(END));for(i=1,q,x=(x^4+2*x^2+1)/(4*(x^3-x));y=lift(x);if(y<(W+1)/2,print(i," ",y," ",lift(x-104-90*kronecker(q,3))),print(i," ",-(W-y)," ",lift(x-104-90*kronecker(q,3))));if(x-104-90*kronecker(q,3)==Mod(0,W),print("Prime")))

%	\ifenfrancais
%	\begin{theorem}[Kok Seng Chua - \`ala LLT - Chebyshev Polynomials - 10/2021]
%	\ 
%	%\newline
%	Si $N_p = \frac{ 2^{\scriptstyle p} + 1}{3}$ ($p$ premier) est premier, alors :
%	$s_{\,p-1} \equiv 5 + 9 \LLegendre{p}{3}  \ \PMod{N_p}$ ,
%	avec : \ $s_{\,0} = 4$ \ et \ $s_{\,n+1} = S_{\,n}^{\,\,2} - 2$ \ pour \ $n=1,2,3, ... \ p-1$ .
%	\end{theorem}
%	\else
%	\begin{theorem}[Kok Seng Chua - \`ala LLT - Chebyshev Polynomials - 10/2021]
%	\ 
%	%\newline
%	If $N_p = \frac{ 2^{\scriptstyle p} + 1}{3}$ ($p$ prime) is a prime, then :
%	$s_{\,p-1} \equiv 5 + 9 \LLegendre{p}{3}  \ \PMod{N_p}$ ,
%	with: \ $s_{\,0} = 4$ \ et \ $s_{\,n+1} = S_{\,n}^{\,\,2} - 2$ \ for \ $n=1,2,3, ... \ p-1$ .
%	\end{theorem}
%	\fi


\vspace{-.2in}



\ifenfrancais
\section{ DiGraph sous $x^2-2$ }
\else
\section{ DiGraph under $x^2-2$ }
\fi

\ifenfrancais
Vasiga \& Shallit (\cite{Shallit-Vasiga}: \emph{On the iteration of certain quadratic maps over GF(p)}) ont \'etudi\'e le DiGraph sous $x^2-2$ modulo un nombre de Mersenne ou un nombre de Fermat et ont montr\'e que,  pour chacun de ces types de nombres, leur DiGraph est constitu\'e d'un unique Arbre G\'eant et de nombreux Cycles.
\else
Vasiga \& Shallit (\cite{Shallit-Vasiga}: \emph{On the iteration of certain quadratic maps over GF(p)}) studied the DiGraph under $x^2-2$ modulo a Mersenne and a Fermat number, proving that, for both kinds of numbers, their DiGraph is made of one Big Tree and of many Cycles.
\fi

\ifenfrancais
\subsection{ ...  modulo un nombre de Mersenne }
\else
\subsection{ ...  modulo a Mersenne number }
\fi

\ifenfrancais
Le test de primalit\'e LLT pour les nombres de Mersenne ($2^q-1$, $q$ premier) parcourt l'Arbre G\'eant du DiGraph sous $x^2-2$ modulo un nombre de Mersenne, en utilisant comme graine (seed) l'une des 3 graines universelles (4, 10, 2/3) et aboutissant \`a $0$ apr\`es $q-1$ it\'erations.
\else
The LLT test for Mersenne numbers ($2^q-1$, $q$ prime) makes use of the Big Tree of the DiGraph under $x^2-2$ modulo a Mersenne number, starting from one of the 3 universal seeds (4, 10, 2/3) and then reaching $0$ after $q-1$ steps.
\fi

\vspace{-.1in}

\ifenfrancais
\begin{theorem}[Test de Primalit\'e de Lucas-Lehmer pour les nombres de Mersenne]
\ 
$M_q = 2^q-1$ (avec $q$ prime) est premier ssi $S_{\,q-1} \equiv 0 \ \PMod{M_q}$ ,
avec : \ $S_{\,0} = 4$ \ et \ $S_{\,i} = S_{\,i-1}^{\,\,2} - 2$ \ pour \ $i=1,2,3, ...$ .
\end{theorem}
\else
\begin{theorem}[Lucas-Lehmer Primality Test for Mersenne numbers]
\ 
$M_q = 2^q-1$ (with $q$ prime) is a prime iff $S_{\,q-1} \equiv 0 \ \PMod{M_q}$ ,
with: \ $S_{\,0} = 4$ \ and \ $S_{\,i} = S_{\,i-1}^{\,\,2} - 2$ \ for \ $i=1,2,3, ...$ .
\end{theorem}
\fi

\ifenfrancais
J'ai trouv\'e la formule donnant le nombre de Cycles d'un tel DiGraph sous $x^2-2$ :
\else
I had found and provided the formula generating the number of Cycles of such a DiGraph under $x^2-2$ :
\fi

\vspace{-.1in}

\ifenfrancais
\begin{theorem}[Reix - ZetaX (du forum \emph{Art of Problem Solving}) - 2]
\ 
Le nombre de Cycles de longueur $L$ (o\`u $L$ divise $q-1=2^su$) du DiGraph $G_{x \rightarrow x^2-2}$ modulo un nombre de Mersenne premier $2^q-1$
est : \vspace{-.15in}
$$\varsigma (L) = \frac{1}{L} \left( \sum_{d \mid L} \mu
\left( \frac{L}{d} \right) 2^d - \sum_{2^s \mid d \mid L} \mu \left(
\frac{L}{d} \right) 2^{d-1} \right)$$
\end{theorem}
\else
\begin{theorem}[Reix - ZetaX (from \emph{Art of Problem Solving} forum) - 2]
\ 
The number of cycles of length $L$ ($L$ divides $q-1=2^su$) in the
digraph $G_{x \rightarrow x^2-2}$ modulo a Mersenne prime $2^q-1$
is: \vspace{-.15in}
$$\varsigma (L) = \frac{1}{L} \left( \sum_{d \mid L} \mu
\left( \frac{L}{d} \right) 2^d - \sum_{2^s \mid d \mid L} \mu \left(
\frac{L}{d} \right) 2^{d-1} \right)$$
\end{theorem}
\fi


\ifenfrancais
\subsection{ ...  modulo un nombre de Fermat }
\else
\subsection{ ...  modulo a Fermat number }
\fi

\ifenfrancais
Un test de primalit\'e pour les nombres de Fermat ($F_n = 2^{2^n}+1$) bas\'e sur la m\'ethode LLT utilise l'Arbre G\'eant du DiGraph sous $x^2-2$ modulo un nombre de Fermat, commen\c{c}ant avec la graine universelle $5$ et atteignant $0$ apr\`es $2^n-2$ it\'erations. Il existe 4 preuves d'un test similaire pour les nombres de Fermat, le mien inclus. On ne sait pas s'il existe d'autres graines universelles. Il est possible de prouver le test de P\'epin au moyen de la technique \`a base d'une S\'equence de Lucas-Lehmer.
\else
A LLT-based primality test for Fermat numbers ($F_n = 2^{2^n}+1$) makes use of the Big Tree of the DiGraph under $x^2-2$ modulo a Fermat number, starting from the universal seed $5$ and then reaching $0$ after $2^n-2$ steps. There exist 4 proofs of a similar primality test for Fermat numbers, including mine. It is not known if there exist several universal seeds. And the Pépin's test can be proved by means of the Lucas-Lehmer Sequence technic.
\fi

\ifenfrancais
\begin{theorem}[Lucas-Lehmer-Reix : Test de Primalit\'e pour les nombres de Fermat]
\ 
\newline
$F_n = 2^{2^n}+1 \ (n \geq 1)$ est premier ssi \ $S_{\,2^n-2} \equiv 0 \ \PMod{F_n}$ ,
avec : \ $S_{\,0} = 5$ \ et \ $S_{\,i} = S_{\,i-1}^{\,\,2} - 2$ \ pour \ $i=1,2,3, ...$ .
\end{theorem}
\else
\begin{theorem}[Lucas-Lehmer-Reix Primality Test for Fermat numbers]
\ 
\newline
$F_n = 2^{2^n}+1 \ (n \geq 1)$ is a prime iff \ $S_{\,2^n-2} \equiv 0 \ \PMod{F_n}$ ,
with: \ $S_{\,0} = 5$ \ and \ $S_{\,i} = S_{\,i-1}^{\,\,2} - 2$ \ for \ $i=1,2,3, ...$ .
\end{theorem}
\fi

\ifenfrancais
\subsection{ Utilisation d'un Cycle \`a la place de l'Arbre G\'eant }
\else
\subsection{ Use of a Cycle instead of the Big Tree }
\fi

\ifenfrancais
On conjecture qu'il est possible de prouver qu'un nombre de Mersenne ou de Fermat est premier en utilisant un Cycle du DiGraph sous $x^2-2$ plut\^ot qu'en utilisant une branche de l'Arbre G\'eant du DiGraph.

Pour le moment, il a \'et\'e uniquement possible de prouver que, si un nombre de Mersenne ou de Fermat est premier, alors son DiGraph pr\'esente des Cycles particuliers qui permettent de g\'en\'erer des tests PRP.
\else
It is conjectured that it is possible to prove that a Mersenne or a Fermat number is prime by using a Cycle of such a DiGraph under $x^2-2$ rather than using a branch of the Big Tree.

However, for now, it has only been possible to prove that, if a Mersenne or a Fermat number is a prime, then their DiGraph contains some characteristic cycles, leading to proven PRP tests.
\fi

\ifenfrancais
\begin{theorem}[Lucas-Lehmer-Reix : PRP Test pour les nombres de Mersenne]
\ 
Si $M_q = 2^q-1$ (avec $q$ premier) est premier, alors $S_{\,q-1} \equiv S_0 \ \PMod{M_q}$ ,
avec : \ $S_{\,0} = 3^2+1/3^2$ \ et \ $S_{\,i} = S_{\,i-1}^{\,\,2} - 2$ \ pour \ $i=1,2,3, ...$ .
\end{theorem}
\else
\begin{theorem}[Lucas-Lehmer-Reix : PRP Test for Mersenne numbers]
\ 
If $M_q = 2^q-1$ (with $q$ prime) is a prime, then $S_{\,q-1} \equiv S_0 \ \PMod{M_q}$ ,
with: \ $S_{\,0} = 3^2+1/3^2$ \ and \ $S_{\,i} = S_{\,i-1}^{\,\,2} - 2$ \ for \ $i=1,2,3, ...$ .
\end{theorem}
\fi

\ifenfrancais
\begin{theorem}[Lucas-Lehmer-Gerbicz : PRP Test pour les nombres de Fermat]
\ 
Si $F_n = 2^{2^n}+1$ est premier, alors $S_{\,2^n-1} \equiv S_0 \ \PMod{F_n}$ ,
avec : \ $S_{\,0} = 1/4$ \ et \ $S_{\,i} = S_{\,i-1}^{\,\,2} - 2$ \ pour \ $i=1,2,3, ...$ .
\end{theorem}
\else
\begin{theorem}[Lucas-Lehmer-Gerbicz PRP Test for Fermat numbers]
\ 
If $F_n = 2^{2^n}+1$ is a prime, then $S_{\,2^n-1} \equiv S_0 \ \PMod{F_n}$ ,
with: \ $S_{\,0} = 1/4$ \ and \ $S_{\,i} = S_{\,i-1}^{\,\,2} - 2$ \ for \ $i=1,2,3, ...$ .
\end{theorem}
\fi


\ifenfrancais
\subsection{ ... modulo un nombre de Wagstaff }
\else
\subsection{ ... modulo a Wagstaff number }
\fi

\ifenfrancais
Comme le DiGraph sous $x^2-2$ modulo un nombre de Wagstaff premier est constitu\'e (preuve par JF Michon, communication priv\'ee) uniquement de Cycles, la seule solution possible pour construire un test de primalit\'e consisterait \`a utiliser un Cycle d'un tel DiGraph. 
\else
Since the DiGraph under $x^2-2$ modulo a Wagstaff number is made (proof by Michon JF, private communication) only of Cycles, it would be only possible to use a Cycle of such a DiGraph for building a Primality Proof.
\fi

\vspace{.05in}
\ifenfrancais
De plus, comme un nombre de Wagstaff n'est pas smooth, il est impossible d'utiliser la technique des S\'equences de Lucas-Lehmer (Lehmer, Ribenboim, ou HC Williams) pour prouver qu'un nombre de Wagstaff est premier. Avec cette technique, on ne peut construire que des tests de PRP.
\else
Moreover, since a Wagstaff number is not smooth, it is impossible to use the Lucas-Lehmer Sequence technic (Lehmer, Ribenboim, or HC Williams) for proving that a Wagstaff number is prime. Only PRP tests are possible with such a technic.
\fi

\vspace{.1in}

\ifenfrancais
J'ai calcul\'e exp\'erimentalement la longueur des Cycles (et le nombre de tels Cycles) du DiGraph sous $x^2-2$ modulo un nombre de Wagstaff avec $q \leq 31$. Ce qui a montr\'e un lien entre le nombre de Cycles de longueur $q-2$ et $q-1$ avec les polyn\^omes irr\'eductibles (s\'equence A165961 de OEIS), quand $W_q$ est premier.
\else
I have experimently computed the length of the Cycles (and the number of such Cycles) of the DiGraph under $x^2-2$ modulo a Wagstaff number with $q \leq 31$, showing a link betwen the number of cycles of length $q-2$ and $q-1$ with irreductible polynoms (sequence A165921 of OEIS) when $W_q$ is a prime.
\fi

\vspace{.05in}
\ifenfrancais
Voir : \cite{Michon-Ravache}: \emph{On different families of invariant irreducible polynomials over $\mathbb{F}_2$}, colonne h6 du tableau en page 173.
\else
See: \cite{Michon-Ravache}: \emph{On different families of invariant irreducible polynomials over $\mathbb{F}_2$}, column h6 of table page 173.
\fi

\vspace{.1in}
%\newpage

\ifenfrancais
\textbf{Donn\'ees exp\'erimentales pour Wagstaff $q \leq 31$}:
\else
\textbf{Experimental data for Wagstaff numbers $q \leq 31$}:
\fi
\vspace{-.15in}
%\footnotesize
\small
%\else
%     L         N         OEIS
%   Length    Number     A165921
% of cycles  of cycles    a(L)
%           of length L
%\fi
\begin{verbatim}
     L          N           OEIS
  Longueur    Nombre       A165921
des cycles  de cycles       a(L)
           de longueur L
----------------------------------
q=7 :
     1          2
     3          1
     5          1             1
     6          1             1
q=11 :
     1          2
     3          1
     5          4
     9          9             9
    10         15            15
q=13 :
     1          2
     2          1
     3          1
     4          1
     6          6
    11         31            31
    12         53            53
q=17 :
     1          2
     2          1
     4          2
     5          3
     8         20
    15        363           363
    16        672           672
q=19 :
     1          2
     3          2
     6          4
     9         37
    17       1285          1285
    18       2407          2407
q=23 :
     1          2
     7          9
    11        124
    21      16641         16641
    22      31713         31713
q=29 :
     1          4
     2          6
     3          2
     4          4
     6          1
    12          9
    14         22
    28         24       1597440
    42          1
    84          2
   363         18
   726          9
  1452          9
  5082          9
 10164        198
 21665          2
 43330          8
 86660         23
129990         18
259980         46
q=31 :
     1          2
     3          1
     5          6
     6          1
    10         48
    15       1454
    29    3085465      3085465
    30    5964488      5964488
\end{verbatim}
\normalsize



\ifenfrancais
\section{ Courbes Elliptiques pour la Preuve de Primalit\'e }
\else
\section{ Elliptic Curves for Primality Proving }
\fi

\ifenfrancais
\subsection {CE pour PP des nombres de Fermat }
\else
\subsection { EC for PP of Fermat numbers }
\fi

\ifenfrancais
En 2007-2008, 2 articles ont d\'emontr\'e qu'il est possible de construire une Preuve de Primalit\'e (PP) pour les nombres de Fermat ($ F_n = 2^{2^n}+1$) en utilisant la m\'ethode des Courbes Elliptiques (CE), faisant suite \`a un article de Benedict Gross en 2005 qui avait montr\'e qu'il est possible de construite une Preuve de Primalit\'e des nombres de Mersenne au moyen des Courbes Elliptiques.
\else
In 2007-2009, 2 papers shown that it is possible to build a Primality Proof (PP) test for the Fermat numbers ($ F_n = 2^{2^n}+1$) by using an Elliptic Curve (EC), following a previous paper by Benedict Gross in 2005 showing that it is possible to build a Primality Proof for Mersenne numbers using a EC.
\fi

\textbf{Robert Denomme \& Gordan Savin} (2007-2008):
Elliptic curve primality tests for Fermat and related primes.  \cite{Denomme-Savin}
 
\textbf{Yu Tsumura} (2009):
Primality tests for Fermat numbers  and $2^{2k+1} \pm 2^{k+1} + 1$ . \cite{Tsumura}

\vspace{.1in}

\ifenfrancais
Le test prouv\'e par \textbf{Tsumura} (en fait un test plus g\'en\'erique, utilisant une variable $m$) appara\^it \`a la page 7 de son article :
\else
The test by \textbf{Tsumura} (with a more generic test, using $m$) appears at page 7 of his paper:
\fi

\vspace{.04in}
%	$ x_{j+1} = \frac{\displaystyle x_j^4+2x_j^2+1}{\displaystyle 4(x_j^3-x_j)}$

$T(x)= \frac{\displaystyle x^4+2x^2+1}{\displaystyle 4(x^3-x)} , \   x_0=5 \ , \   x_{j+1} = T(x_j) $

\vspace{.02in}
\ifenfrancais
$\text{Si } \ x_{2^{k-1}-1} \equiv \pm 1 \ \pmod{F_k} \ , \ \text{alors } F_k \text{ est premier} .$
\else
$\text{If } \ x_{2^{k-1}-1} \equiv \pm 1 \ \pmod{F_k} \ , \ \text{then } F_k \text{ is prime} .$
\fi

\vspace{.1in}

\ifenfrancais
Le test prouv\'e par \textbf{Denomme \& Savin} appara\^it au chapitre 4 \`a la page 7 (ou 2404) comme :
\else
The test by \textbf{Denomme \& Savin} appears in Chapter 4 at page 7 (or 2404) as:
\fi

\vspace{.04in}
$x_1 = 5 \ , \ x_{m+1} = 1/2 \left( \frac{\displaystyle x_m}{\displaystyle i} + \frac{\displaystyle i}{\displaystyle x_m} \right) $

\vspace{.02in}
\ifenfrancais
$F_n  \text{ est premier ssi } x_{2^k} \  \equiv \ 0 \ \pmod{2^{2^k}+i}$ .
\else
$F_n  \text{ is prime iff } x_{2^k} \  \equiv \ 0 \ \pmod{2^{2^k}+i}$ .
\fi

\vspace{.1in}

\ifenfrancais
Bien que ce test semble assez bizarre, et parce que $x_{2j}$ est une fraction o\`u le nombre imaginaire $i$ n'appara\^it pas, il est possible de transformer le test (en exprimant $x_{2n}$ en fonction de $x_{2(n-1)}$) dans une forme presque identique au test de Tsumura's, avec un coefficient $S = \pm 1$ , puisque $T(-x) = - T(x)$ .
\else
Though this test looks weird, and since $x_{2j}$ is a fraction where the imaginary number $i$ does not appear, it is possible to transform the test (expressing $x_{2n}$ in function of $x_{2(n-1)}$)  in nearly exact Tsumura's test, with a coefficient $S = \pm 1$ , since $T(-x) = - T(x)$ .
\fi

\vspace{.1in}

\ifenfrancais
Aussi, nous ne consid\'erons i\c{c}i que le test de primalit\'e suivant :
\else
So, we will only consider the following EC-based Primality Test for Fermats:
\fi

\ifenfrancais
\begin{theorem}[Denomme/Savin - Tsumura = DST]
\ 
\vspace{.04in}

$dst(x)= \frac{\displaystyle x^4+2x^2+1}{\displaystyle 4(x^3-x)} , x_1=F_1=5, x_{j+1} = dst(x_j)$.

\vspace{.05in}
$\text{Si } \ x_{2^{n-1}} \equiv -1 \ \pmod{F_n} \ , \ \text{alors } F_n  \text{ est premier} $ .

\end{theorem}
\else
\begin{theorem}[Denomme/Savin - Tsumura = DST]
\ 
\vspace{.04in}

$x_1=F_1=5 \ , \   x_{j+1} = \frac{\displaystyle x_j^4+2x_j^2+1}{\displaystyle 4(x_j^3-x_j)}$ .

\vspace{.05in}
$\text{If } \ x_{2^{n-1}} \equiv -1 \ \pmod{F_n} \ , \ \text{then } F_n  \text{ is prime} $ .

\end{theorem}
\fi


\ifenfrancais
\subsection { Code en Pari/gp code pour DST }
\else
\subsection { Pari/gp code for DST }
\fi

\ifenfrancais
4 tests diff\'erents sont fournis. Utiliser une ligne x1=... pour utiliser le test de votre choix)
\else
(4 tests are provided. Just use the x1=... line you want to test)
\fi

\small
\begin{verbatim}
ECPPforFermat(n,p)=
{
F=2^(2^n)+1;
if(p>0,printf("-1/3: %10d %10d\n",
      lift(Mod(-1/3,F)),-lift(Mod(1/3,F))));
x1=Mod( 5,F); iF=2^(n-1); xF=Mod(-1,F);
x1=Mod(-5,F); iF=2^(n-1); xF=Mod( 1,F);
x1=Mod( 4,F); iF=2^(n-1); xF=Mod( 0,F);
x1=Mod( 3,F); iF=2^n-1;   xF=Mod(-1/3,F);
x=x1;
for(i=2,iF,
        x=(x^4+2*x^2+1)/(4*x*(x^2-1));
        if(p==1,print(lift(x)));
);
if(p==0,print(lift(x)));
if(x == xF,
    printf("2^2^%d is prime !\n\n", n);
  , printf("2^2^%d is composite !\n\n", n); );
}
ECPPforFermat(4,1);
for(n=2,10,ECPPforFermat(n,0);print(" "));
\end{verbatim}
\normalsize


\ifenfrancais
\subsection { DiGraph sous $\frac{\displaystyle x^4+2x^2+1}{\displaystyle 4(x^3-x)}$ modulo un nombre de Fermat }
\else
\subsection { DiGraph under $\frac{\displaystyle x^4+2x^2+1}{\displaystyle 4(x^3-x)}$ modulo a Fermat }
\fi

\ifenfrancais
Le DiGraph sous $\frac{\displaystyle x^4+2x^2+1}{\displaystyle 4(x^3-x)}$ modulo un nombre de Fermat semble (\`a partir des 4 seuls nombres de Fermat connus) avoir la structure suivante :
\else
The DiGraph under $\frac{\displaystyle x^4+2x^2+1}{\displaystyle 4(x^3-x)}$ modulo a Fermat number seems (based on only the 4 known Fermat primes) to be made of:
\fi

\vspace{.05in}
\ifenfrancais
\ - 3 \textbf{Arbres G\'eants} finissant par : $-1, +1, \text{ ou } 0$, de hauteur $2^{n-1}$ avec $2\sum_{i=0}^{n-1}4^i$ n\oe uds,
\else
\ - 3 \textbf{Big Trees} ending at: $-1, +1, \text{ and } 0$, of heigth $2^{n-1}$ with $2\sum_{i=0}^{n-1}4^i$ nodes,
\fi

\vspace{.05in}
\ifenfrancais
\  -  \emph{de nombreux} \textbf{Cycles}, dont $2^{n-1}+1$ cycles de longueur $2^n-2$ avec de petits arbres (de longueur 1) de 3 n\oe uds attach\'es.
\else
\  -  plus \emph{plenty} ($2^{n-1}+1 ?$) of \textbf{Cycles} of length $2^n-2$ with small trees (length 1) of 4 nodes attached.
\fi


\ifenfrancais
\subsubsection { Arbres G\'eants }
\else
\subsubsection { Big Trees }
\fi

\ifenfrancais
Le  \textbf{premier Arbre G\'eant} (finissant par : $-1$) est celui utilis\'e par le test de primalit\'e DST ci-dessus (th\'eor\`eme 13).
\else
The \textbf{first Big Tree} (ending by: $-1$) deals with the above proven DST Primality Test (theorem 13).
\fi

\ifenfrancais
Example pour $n=4$ : $\PMod{F_{4}}$ :
\else
Example for $n=4$ : $\PMod{F_{4}}$ :
\fi

$x_1 = 5
\stackrel{2}{\mapsto} \text{-9283}
\stackrel{3}{\mapsto} \text{25064}
\stackrel{4}{\mapsto} \text{-26225}
\stackrel{5}{\mapsto} \text{-25143}
\stackrel{6}{\mapsto} \text{-3300}
\stackrel{7}{\mapsto} \text{4079}
\stackrel{8}{\mapsto} \textbf{-1}$

\vspace{.1in}

\ifenfrancais
Le \textbf{second Arbre G\'eant} (finissant par : $+1$) est similaire au premier Arbre G\'eant, avec un coefficient $-1$ appliqu\'e \`a tous les n\oe uds.
\else
The \textbf{second Big Tree} (ending by: $+1$) is similar to the first Big Tree, with a $-1$ coefficient applied to all nodes.
\fi

\vspace{.1in}

\ifenfrancais
Le \textbf{troisi\`eme Arbre G\'eant} (finissant par : $0$) est associ\'e \`a un test candidat de primalit\'e tr\`es proche du test LLT pour les nombres de Mersenne (m\^eme graine $4$ et m\^eme test final avec $0$).
\else
The \textbf{third Big Tree} (ending by: $0$) is associated with a candidate primality test very close to the LLT test for Mersennes (same seed $4$ and same final test with $0$).
\fi

\ifenfrancais
Example pour $n=4$ : $\PMod{F_{4}}$ :
\else
Example for $n=4$ : $\PMod{F_{4}}$ :
\fi

$x_1 = 4
\stackrel{2}{\mapsto} \text{-4641}
\stackrel{3}{\mapsto} \text{-14136}
\stackrel{4}{\mapsto} \text{17727}
\stackrel{5}{\mapsto} \text{-5367}
\stackrel{6}{\mapsto} \text{-10395}
\stackrel{7}{\mapsto} \text{-256}
\stackrel{8}{\mapsto} \textbf{0}$

\vspace{.05in}

\ifenfrancais
\begin{conjecture}[Denomme/Savin - Tsumura - Reix]
\ 
\vspace{.04in}

$x_1=\pm 4 \ , \   x_{j+1} = \frac{\displaystyle x_j^4+2x_j^2+1}{\displaystyle 4(x_j^3-x_j)}$ .

\vspace{.05in}
$\text{Si } \ x_{2^{n-1}} \equiv 0 \ \pmod{F_n} \ , \ \text{ alors } F_n  \text{ est premier} $ .
\end{conjecture}
\else
\begin{conjecture}[Denomme/Savin - Tsumura - Reix]
\ 
\vspace{.04in}

$x_1=\pm 4 \ , \   x_{j+1} = \frac{\displaystyle x_j^4+2x_j^2+1}{\displaystyle 4(x_j^3-x_j)}$ .

\vspace{.05in}
$\text{If } \ x_{2^{n-1}} \equiv 0 \ \pmod{F_n} \ , \ \text{then } F_n  \text{ is prime} $ .
\end{conjecture}
\fi

%	Works also with x_1 = (F-1)/4
%
%	? n=4;F=2^2^n+1;END=Mod(0,F);x=Mod(4,F);print(0," ",lift(END));for(i=2,2^(n-1),x=(x^4+2*x^2+1)/(4*(x^3-x));y=lift(x);if(y<(F+1)/2,print(y),print(-(F-y))));if(x==END,print("Prime"))
%	0 0
%	-4641
%	-14136
%	17727
%	-5367
%	-10395
%	-256
%	0
%	Prime


\subsubsection { Cycles }


\ifenfrancais
Maintenant, si l'on \'etudie les nombreux Cycles du DiGraph sous $\frac{\displaystyle x^4+2x^2+1}{\displaystyle 4(x^3-x)}$ modulo un nombre de Fermat, l'un d'entre eux est sp\'ecial (voir le chapitre suivant sur les nombres de Wagstaff) et laisse imaginer le test de Primalit\'e (ou simplement PRP) pour les nombres de Wagstaff :
\else
Now, looking at the numerous Cycles of the DiGraph under $\frac{\displaystyle x^4+2x^2+1}{\displaystyle 4(x^3-x)}$ modulo a Fermat number, one is special (see next section for Wagstaff numbers) and leads to the candidate PRP/Primality test for Wagstaff numbers:
\fi


\ifenfrancais
\begin{conjecture}[Denomme/Savin - Tsumura - Reix]
\ 
\vspace{.02in}

$x_1=F_0=3 \ , \   x_{j+1} = \frac{\displaystyle x_j^4+2x_j^2+1}{\displaystyle 4(x_j^3-x_j)}$ .

$\text{Si } \ x_{2^{n}-1} \equiv -1/3 \ \pmod{F_n} \ , \ \text{alors } F_n \text{ est premier} $ .
\end{conjecture}
\else
\begin{conjecture}[Denomme/Savin - Tsumura - Reix]
\ 
\vspace{.02in}

$x_1=F_0=3 \ , \   x_{j+1} = \frac{\displaystyle x_j^4+2x_j^2+1}{\displaystyle 4(x_j^3-x_j)}$ .

$\text{If } \ x_{2^{n}-1} \equiv -1/3 \ \pmod{F_n} \ , \ \text{then } F_n \text{ is prime} $ .
\end{conjecture}
\fi


% Equivalent à partir de -1/3 et finir à -1/3 en 2^n-2 étapes
% 
%	? n=4;F=2^2^n+1;END=Mod(-1/3,F);x=END;print(0," ",lift(END));for(i=2,2^n-1,x=(x^4+2*x^2+1)/(4*(x^3-x));y=lift(x);if(y<(F+1)/2,print(y),print(-(F-y))));if(x==END,print("Prime"))
%	0 43691   = -21846
%	19116
%	5433
%	17830
%	3117
%	4769
%	23216
%	21846
%	-19116
%	-5433
%	-17830
%	-3117
%	-4769
%	-23216
%	-21846
%	Prime
%	
%	? -(F-43691)
%	%64 = -21846


%\vspace{.05in}

\ifenfrancais
(\emph{La graine $3$ ne fait pas partie du Cycle. Mais cette conjecture est \'equivalente \`a partir de $x_1=-1/3 \pmod{F_n}$ et y revenir apr\`es $2^n-2$ \'etapes.})
\else
(\emph{The seed $3$ is out of the Cycle. However this conjecture is the same as starting from $x_1=-1/3 \pmod{F_n}$ and come back to it after $2^n-2$ steps.})
\fi

\ifenfrancais
Example pour $n=3$:
\else
Example for $n=3$:
\fi

$\PMod{F_{3}}$ $x_1 = 3
\stackrel{2}{\mapsto} \text{76}
\stackrel{3}{\mapsto} \text{108}
\stackrel{4}{\mapsto} \text{86}
\stackrel{5}{\mapsto} \text{-76}
\stackrel{6}{\mapsto} \text{-108}
\stackrel{7}{\mapsto} \text{-86} = \textbf{-1/3}
\stackrel{}{\mapsto} \text{76} \dots $

\ifenfrancais
Example pour $n=4$:
\else
Example for $n=4$:
\fi

%	? n=4;F=2^2^n+1;END=Mod(-1/3,F);x=END;print(-(F-lift(END)));for(i=2,2^n-1,x=(x^4+2*x^2+1)/(4*(x^3-x));y=lift(x);if(y<(F+1)/2,print(y),print(-(F-y))));if(x==END,print("Prime"))

$\PMod{F_{4}}$ $x_1 = \textbf{-1/3} = -21846
\stackrel{2}{\mapsto} \text{19116}
\stackrel{3}{\mapsto} \text{5433}
\stackrel{4}{\mapsto} \text{17830}
\stackrel{5}{\mapsto} \text{3117}
\stackrel{6}{\mapsto} \text{4769}
\stackrel{7}{\mapsto} \text{23216}
\stackrel{8}{\mapsto} \text{21846}
\stackrel{9}{\mapsto} \text{-19116}
\stackrel{10}{\mapsto} \text{-5433}
\stackrel{11}{\mapsto} \text{-17830}
\stackrel{12}{\mapsto} \text{-3117}
\stackrel{13}{\mapsto} \text{-4769}
\stackrel{14}{\mapsto} \text{-23216}
\stackrel{15}{\mapsto} \text{-21846} = \textbf{-1/3}$

\ifenfrancais
\subsection { CE pour PP des nombres de Wagstaff }
\else
\subsection { EC for PP of Wagstaff numbers }
\fi

\ifenfrancais
\subsubsection { DiGraph sous $\frac{\displaystyle x^4+2x^2+1}{\displaystyle 4(x^3-x)}$ modulo un nombre de Wagstaff }
\else
\subsubsection { DiGraph under $\frac{\displaystyle x^4+2x^2+1}{\displaystyle 4(x^3-x)}$ modulo a Wagstaff }
\fi


\ifenfrancais
Si l'on regarde le DiGraph sous $\frac{\displaystyle x^4+2x^2+1}{\displaystyle 4(x^3-x)}$ modulo un nombre de Wagstaff $W_q$ premier, il appara\^it exp\'erimentalement (pour les premi\`eres valeurs de $q$ qui permettent d'\'etudier le DiGraph: $q \leq 31$) qu'il est constitu\'e uniquement de Cycles de longeur L telle que $L \mid q-2$ avec 4 n\oe uds attach\'es \`a chaque n\oe ud de chaque Cycle.
\else
Looking at the DiGraph under $\frac{\displaystyle x^4+2x^2+1}{\displaystyle 4(x^3-x)}$ modulo a Wagstaff number $W_q$, it appears experimentaly (for the first values of $q$ that enable to study the DiGraph: $q \leq 31$) that it is made only of Cycles of length L such that $L \mid q-2$ with 4 nodes attached to each node of each cycle.
\fi

\vspace{.05in}
\ifenfrancais
Il y a toujours (et parfois uniquement, quand $q-2$ est premier) des Cycles de longueur $q-2$ ; et le nombre de ces Cycles est $2 \times a(q-2)$ o\`u $a(n)$ vient de la s\'erie A165921 d'OEIS (polyn\^omes irr\'eductibles).
\else
There are always (and sometimes only) cycles of length $q-2$ and the number of these cycles is $2 \times a(q-2)$ where $a(n)$ comes from the OEIS A165921 series (irreductible polynomials).
\fi

\vspace{.05in}
\ifenfrancais
Il est remarquable que la relation entre le nombre de Cycles du DiGraph modulo un nombre de Wagstaff et les polyn\^omes irr\'eductibles apparaisse \`a la fois avec 
$x^2-2$ et avec $\frac{ x^4+2x^2+1}{ 4(x^3-x)}$.
\else
It is noticeable that the relationship between the length of Cycles of the DiGraph modulo Wagstaff numbers and with the irreductible polynomials appears both with $x^2-2$ and with $\frac{ x^4+2x^2+1}{ 4(x^3-x)}$.
\fi

\vspace{.2in}
\ifenfrancais
\textbf{Donn\'es exp\'erimentales pour les  Wagstaff $q \leq 31$}:
\else
\textbf{Experimental data for Wagstaff numbers $q \leq 31$}:
\fi
\vspace{-.2in}
%    L         N         OEIS
%  Length    Number     A165921
%of cycles  of cycles   a(L) = h6(q-2)
%          of length L
\small
\begin{verbatim}
     L          N           OEIS
  Longueur    Nombre       A165921
des cycles  de cycles       a(L)
           de longueur L    
-----------------------------------
q=7  :
     5          2             1
q=11 :
     1          2
     3          2     
     9         18             9
q=13 :
    11         62            31
q=17 :
     1          1
     5          6     
    15        726           363
q=19 :
    17       2570          1285
q=23 :
     1          1
     7         18     
    21      33282         16641
q=29:
     1          6
     2         68
     4         72
     6         30
    12        300
   324         28
   648       8176
  1184         56
  2368        112
q=31 :
    29    6170930       3085465 
\end{verbatim}
\normalsize

\ifenfrancais
Concernant $q=29$, les 6 Cycles de longueur 1 sont :
                  62409100,
                  68475438,
                175217690,
                110481533,
                116547871,
                176629914.
%Et, comme $W_{29}=59 * 3033169$, le nombre de $n$ tel que $1/(4(n^3-n)) \PMod{W_{29}}$ n'existe pas est $18199176$ for those $= 59k\pm1$ or $=59k$, and $543$ for those $= 3033169k\pm1$ or $=3033169k$.
\else
About $q=29$, the 6 Cycles of length 1 are:
                  62409100,
                  68475438,
                175217690,
                110481533,
                116547871,
                176629914.
%And, since $W_{29}=59 * 3033169$, the number of $n$ such that $1/(4(n^3-n)) \PMod{W_{29}}$ does not exist is $18199176$ for those $= 59k\pm1$ or $=59k$, and $543$ for those $= 3033169k\pm1$ or $=3033169k$.
\fi

\vspace{.1in}
\ifenfrancais
Voici les Longueur et Nombres de Cycles pour $59$ et $3033169$ ($W_{29}=59 * 3033169$). ($N$ pour $L \geq 6$ de $3033169$ divise les $N$ correspondants de $W_{29}$.)
\else
Here are the Lengths and Number of cycles for $59$ and $3033169$ ($W_{29}=59 * 3033169$). ($N$ for $L \geq 6$ of $3033169$ divides the corresponding $N$ of $W_{29}$.)
\fi

\vspace{-.1in}
%    L         N         
%  Length    Number     
%of cycles  of cycles   
%          of length L
\small
\begin{verbatim}
     L          N           
  Longueur    Nombre      
des cycles  de cycles      
           de longueur L    
---------------------
59:
      1         2
      2         8
      4         2
3033169:
      1         2
      2         4
      4         4
      6         5
     12        20
    324         2
    648       584
   1184         4
   2368         8
\end{verbatim}
\normalsize


\ifenfrancais
\subsubsection { Test PRP candidat pour les Wagstaff }
\else
\subsubsection { Candidate PRP test for Wagstaff numbers }
\fi

\ifenfrancais
De plus, en regardant un Cycle sp\'ecifique commen\c{c}ant \`a $3$ et finissant \`a $-1/3$ (identique \`a ce qui a \'et\'e vu pour les nombres de Fermat), il appara\^it que c'est un test PRP candidat.
\else
Morover, looking at a specific Cycle starting at $3$ and ending at $-1/3$ (same as previously seen for Fermat numbers), it appears that it is a candidate PRP test.
\fi

\ifenfrancais
\emph{J'ai v\'erifi\'e que ce test r\'eussit pour tout les $q$ tels que $W_q$ est connu pour \^etre premier (jusqu'\`a $q=141.079$), et qu'il \'echoue pour tous les $q$ (inf\'erieurs \`a $14.479$) tel que $W_q$ n'est pas premier.}
\else
\emph{It has been checked that this test succeeds for all $q$ such that $W_q$ is known to be a prime (up to $q=141,079$), and that it fails for all $q$ (below 14,479) such that $W_q$ is not a prime.}
\fi

\vspace{.05in}

\ifenfrancais
\begin{conjecture}[Denomme/Savin - Tsumura - Reix = DSTR]
\ 

\vspace{-.05in}
\quad \  $x_1=3 \text{ ou } x_1=-1/3 \ , \   x_{j+1} = \frac{\displaystyle x_j^4+2x_j^2+1}{\displaystyle 4(x_j^3-x_j)}$

\vspace{.05in}
\quad \  $\text{Si } \ x_{q-1} \  \equiv \ -1/3 \ \pmod{W_q} \ ,$

\vspace{.05in}
\quad \ \ $ \ \text{alors } W_q = \frac{2^q+1}{3} \text{ est (PRobablement) Premier}$ .

\end{conjecture}
\else
\begin{conjecture}[Denomme/Savin - Tsumura - Reix = DSTR]
\ 

\vspace{-.05in}
\quad \  $x_1=3 \text{ or } x_1=-1/3  \ , \   x_{j+1} = \frac{\displaystyle x_j^4+2x_j^2+1}{\displaystyle 4(x_j^3-x_j)}$

\vspace{.05in}
\quad \  $\text{If } \ x_{q-1} \  \equiv \ -1/3 \ \pmod{W_q} \ ,$

\vspace{.05in}
\quad \ \ $ \ \text{then } W_q = \frac{2^q+1}{3} \text{ is (PRobably) Prime}$ .

\end{conjecture}
\fi


% Equivalent à partir de -1/3 et finir à -1/3 en 2^n-2 étapes

%	? q=11;W=(2^q+1)/3;END=Mod(-1/3,W);x=END;print(0," ",lift(END));for(i=2,q-1,x=(x^4+2*x^2+1)/(4*(x^3-x));y=lift(x);if(y<(W+1)/2,print(i," ",y),print(i," ",-(W-y))));if(x==END,print("Prime"))
%	0 455   = -228
%	1 -312
%	2 181
%	3 130
%	4 112
%	5 111
%	6 -185
%	7 134
%	8 -65
%	9 -228
%	Prime
%	
%	-(W-455) = -228

%	? a=3;b=(a^4+2*a^2+1)/(4*(a^3-a))
%	%66 = 25/24
%	? a=-1/3;b=(a^4+2*a^2+1)/(4*(a^3-a))
%	%47 = 25/24


\vspace{.05in}

$q=\textbf{7}$: $\PMod{W_{7}=43}$ :

$x_1 = 3
\stackrel{2}{\mapsto} \text{10}
\stackrel{3}{\mapsto} \text{-19}
\stackrel{4}{\mapsto} \text{16}
\stackrel{5}{\mapsto} \text{15}
\stackrel{6}{\mapsto} \text{14} = \textbf{-1/3}
\stackrel{}{\mapsto} \text{10} = x_2 \dots $

\vspace{.1in}

$q=\textbf{11}$: $\PMod{W_{11}=683}$ :

$x_1 = 3
\stackrel{2}{\mapsto} \text{-312}
\stackrel{3}{\mapsto} \text{181}
\stackrel{4}{\mapsto} \text{130}
\stackrel{5}{\mapsto} \text{112}
\stackrel{6}{\mapsto} \text{111}
\stackrel{7}{\mapsto} \text{-185}
\stackrel{8}{\mapsto} \text{134}
\stackrel{9}{\mapsto} \text{-65}
\stackrel{10=q-1}{\mapsto} \text{-228} = \textbf{-1/3}
\stackrel{}{\mapsto} \text{-312}= x_2 \dots $

\vspace{.1in}

$q=\textbf{17}$: $\PMod{W_{17}=43691}$ :

$x_1 = 3
\stackrel{2}{\mapsto} \text{-20024}
\stackrel{3}{\mapsto} \text{-4673}
\stackrel{4}{\mapsto} \text{-4921}
\stackrel{5}{\mapsto} \text{17563}
\stackrel{6}{\mapsto} \text{12984}
\stackrel{7}{\mapsto} \text{-5695}
\stackrel{8}{\mapsto} \text{18667}
\stackrel{9}{\mapsto} \text{20891}
\stackrel{10}{\mapsto} \text{-6366}
\stackrel{11}{\mapsto} \text{-13224}
\stackrel{12}{\mapsto} \text{-19227}
\stackrel{13}{\mapsto} \text{-18235}
\stackrel{14}{\mapsto} \text{-15993}
\stackrel{15}{\mapsto} \text{511}
\stackrel{16=q-1}{\mapsto} \text{-14564} = \textbf{-1/3}
\stackrel{}{\mapsto} \text{-20024} = x_2 \dots $

\vspace{.1in}

$q=\textbf{19}$: $\PMod{W_{19}= 174763}$ :

$x_1 = 3
\stackrel{2}{\mapsto} \text{36410}
\stackrel{3}{\mapsto} \text{-62146}
\stackrel{4}{\mapsto} \text{65849}
\stackrel{5}{\mapsto} \text{-57980}
\stackrel{6}{\mapsto} \text{15234}
\stackrel{7}{\mapsto} \text{76579}
\stackrel{8}{\mapsto} \text{76951}
\stackrel{9}{\mapsto} \text{-1581}
\stackrel{10}{\mapsto} \text{34057}
\stackrel{11}{\mapsto} \text{58680}
\stackrel{12}{\mapsto} \text{-25587}
\stackrel{13}{\mapsto} \text{67892}
\stackrel{14}{\mapsto} \text{66223}
\stackrel{15}{\mapsto} \text{56973}
\stackrel{16}{\mapsto} \text{-77064}
\stackrel{17}{\mapsto} \text{1023}
\stackrel{18=q-1}{\mapsto} \text{58254} = \textbf{-1/3}
\stackrel{}{\mapsto} \text{36410} = x_2 \dots $

\vspace{.1in}

$q=\textbf{23}$: $\PMod{W_{23}= 2796203}$ :

$x_1 = 3
\stackrel{2}{\mapsto} \text{-1281592}
\stackrel{3}{\mapsto} \text{-1066801}
\stackrel{4}{\mapsto} \text{-896417}
\stackrel{5}{\mapsto} \text{973270}
\stackrel{6}{\mapsto} \text{-1074287}
\stackrel{7}{\mapsto} \text{1341106}
\stackrel{8}{\mapsto} \text{366351}
\stackrel{9}{\mapsto} \text{333831}
\stackrel{10}{\mapsto} \text{1107490}
\stackrel{11}{\mapsto} \text{937393}
\stackrel{12}{\mapsto} \text{907735}
\stackrel{13}{\mapsto} \text{-1298722}
\stackrel{14}{\mapsto} \text{-300994}
\stackrel{15}{\mapsto} \text{416316}
\stackrel{16}{\mapsto} \text{-572929}
\stackrel{17}{\mapsto} \text{1302116}
\stackrel{18}{\mapsto} \text{67769}
\stackrel{19}{\mapsto} \text{-1258056}
\stackrel{20}{\mapsto} \text{370787}
\stackrel{21}{\mapsto} \text{-4097}
\stackrel{22=q-1}{\mapsto} \text{-932068} = \textbf{-1/3}
\stackrel{}{\mapsto} \text{-1281592} = x_2 \dots $


\vspace{.1in}

$q=\textbf{29}$: $\PMod{W_{29}=178956971}$ :

$x_1 = 3
\stackrel{2}{\mapsto} \text{-82021944}
\stackrel{3}{\mapsto} \text{47279044}
\stackrel{4}{\mapsto} \text{-6769274}
\stackrel{5}{\mapsto} \text{-82432991}
\stackrel{6}{\mapsto} \text{65892000}
\stackrel{7}{\mapsto} \text{-73670956}
\stackrel{8}{\mapsto} \text{25925635}
\stackrel{9}{\mapsto} \text{-6570850}
\stackrel{10}{\mapsto} \text{63682155}
\stackrel{11}{\mapsto} \text{-19930570}
\stackrel{12}{\mapsto} \text{8768966}
\stackrel{13}{\mapsto} \text{-80742700}
\stackrel{14}{\mapsto} \text{-83486737}
\stackrel{15}{\mapsto} \text{60018973}
\stackrel{16}{\mapsto} \text{-89447744}
\stackrel{17}{\mapsto} \text{13157511}
\stackrel{18}{\mapsto} \text{-5229550}
\stackrel{19}{\mapsto} \text{-51905325}
\stackrel{20}{\mapsto} \text{-19198981}
\stackrel{21}{\mapsto} \text{-57252499}
\stackrel{22}{\mapsto} \text{22541219}
\stackrel{23}{\mapsto} \text{11253408}
\stackrel{24}{\mapsto} \text{-29629532}
\stackrel{25}{\mapsto} \text{77141064}
\stackrel{26}{\mapsto} \text{-89199707}
\stackrel{27}{\mapsto} \text{-49038102}
\stackrel{28=q-1}{\mapsto} \text{-16133813} \not = \textbf{-1/3} \  (=119304647)
\stackrel{29}{\mapsto} \dots
\stackrel{2370}{\mapsto} \text{28969859}
\stackrel{2371}{\mapsto} \text{47279044} = x_3$
\ifenfrancais
(Cycle de longueur $2368 = 2^6 \times 37$)
\else
(Cycle of length $2368 = 2^6 \times 37$)
\fi

\vspace{.1in}

$q=\textbf{31}$: $\PMod{W_{31}=715827883}$ :

$x_1 = 3
\stackrel{2}{\mapsto} \text{149130810}
\stackrel{3}{\mapsto} \text{11279171}
\stackrel{4}{\mapsto} \text{-66109836}
\stackrel{5}{\mapsto} \text{249450180}
\stackrel{6}{\mapsto} \text{-280431833}
\stackrel{7}{\mapsto} \text{-97596511}
\stackrel{8}{\mapsto} \text{-332690658}
\stackrel{9}{\mapsto} \text{-329143902}
\stackrel{10}{\mapsto} \text{-88687766}
\stackrel{11}{\mapsto} \text{324996427}
\stackrel{12}{\mapsto} \text{190514966}
\stackrel{13}{\mapsto} \text{-207459777}
\stackrel{14}{\mapsto} \text{131027028}
\stackrel{15}{\mapsto} \text{36447093}
\stackrel{16}{\mapsto} \text{-245289057}
\stackrel{17}{\mapsto} \text{199095424}
\stackrel{18}{\mapsto} \text{27348828}
\stackrel{19}{\mapsto} \text{151062042}
\stackrel{20}{\mapsto} \text{106649512}
\stackrel{21}{\mapsto} \text{-28457251}
\stackrel{22}{\mapsto} \text{232233162}
\stackrel{23}{\mapsto} \text{319560515}
\stackrel{24}{\mapsto} \text{-46286542}
\stackrel{25}{\mapsto} \text{120897033}
\stackrel{26}{\mapsto} \text{167096450}
\stackrel{27}{\mapsto} \text{-279090714}
\stackrel{28}{\mapsto} \text{220510103}
\stackrel{29}{\mapsto} \text{65535}
\stackrel{30=q-1}{\mapsto} \text{238609294}  = \textbf{-1/3}
\stackrel{}{\mapsto} \text{149130810} = x_2 \dots $



\ifenfrancais
\subsection { Code Pari/gp pour DSTR }
\else
\subsection { Pari/gp code for DSTR }
\fi

\small
\begin{verbatim}
ECPPforWagstaff(q,p)=
{
w=(2^q+1)/3;
x1=Mod(3,w); iF=q-1; xF= Mod(-1/3,w);
x=x1;
if(p>0,print("W",q,": ",w));
if(p>1,print("-1/3: ",lift(Mod(-1/3,w))));
for(i=2,iF,
    x=(x^4+2*x^2+1)/(4*x*(x^2-1));
    if(p>0,printf("%3d %20d \n",i,lift(x))); );
if(x == xF, printf("W%d is prime ! \n", q)
    , if(p>0,printf("W%d is composite.\n", q)); );
}
ECPPforWagstaff(7,2);
ECPPforWagstaff(17,2);
forprime(q=19,15000,ECPPforWagstaff(q,0));
\end{verbatim}
\normalsize



\ifenfrancais
\subsection { Quelle est la vitesse de l'algorithme DSTR ? }

\`A vue de nez, ce test semble \^etre environ 5 fois plus lent que le test LLT pour les nombres de Mersenne ou de Fermat. Ce qui reste extr\^emement rapide !
\else
\subsection { How fast is the DSTR algorithm ? }

As a quick look, such a test seems to be about 5 times slower that the LLT test for Mersennes of Fermats. Thus it's still a fast test!
\fi



\ifenfrancais
\section { Images de Digraphs sous  $\frac{\displaystyle x^4+2x^2+1}{\displaystyle 4(x^3-x)}$ modulo ... }
\else
\section { Images of Digraphs under $\frac{\displaystyle x^4+2x^2+1}{\displaystyle 4(x^3-x)}$ modulo ... }
\fi

\subsection { Digraph modulo Fermat $F_3$ }

\includegraphics[scale=0.45]{EllipticCurvePrimalityTestforFermat-List-n3-by-neato}


\ifenfrancais
\subsubsection {Arbre G\'eant  $x_1 = 5 \stackrel{2}{\mapsto} \text{-35} ... \stackrel{8}{\mapsto} \textbf{-1}$}
\else
\subsubsection {Big Tree  $x_1 = 5 \stackrel{2}{\mapsto} \text{-35} ... \stackrel{8}{\mapsto} \textbf{-1}$}
\fi

\includegraphics[scale=0.6]{EllipticCurvePrimalityTestforFermat-List-n3-by-neato-detail5-1}


\subsubsection {Cycle  $x_1 = 3 \stackrel{2}{\mapsto} \text{76} ... \stackrel{}{\mapsto} \textbf{-86} = \textbf{-1/3}$}

\includegraphics[scale=0.5]{EllipticCurvePrimalityTestforFermat-List-n3-by-neato-detail3-86}


\subsection { Digraph modulo Wagstaff $W_7$ }

\includegraphics[scale=0.45]{EllipticCurvePrimalityTestforWagstaff-List-q7-by-neato}


\subsection { Digraph modulo Wagstaff $W_{11}$ }

\includegraphics[scale=0.45]{EllipticCurvePrimalityTestforWagstaff-List-q11-by-neato}


\subsubsection { Cycle $3\stackrel{2}{\mapsto} \text{-312} \stackrel{3}{\mapsto} \text{...} \stackrel{10=q-1}{\mapsto} \text{-228} = \textbf{-1/3}$}

\includegraphics[scale=0.5]{EllipticCurvePrimalityTestforWagstaff-List-q11-by-neato-detail3-65}


\subsection { Digraph modulo Wagstaff $W_{13}$ }

\includegraphics[scale=0.5]{EllipticCurvePrimalityTestforWagstaff-List-q13-by-neato}


\subsubsection { Cycle $3\stackrel{2}{\mapsto} \text{570} \stackrel{3}{\mapsto} \text{...} \stackrel{12=q-1}{\mapsto} \text{910} = \textbf{-1/3}$}

\includegraphics[scale=0.6]{EllipticCurvePrimalityTestforWagstaff-List-q13-by-neato-detail3-}


%\newpage

\bibliographystyle{ieeetr}
\bibliography{QuadratureWagstaffv3}

\end{article}
\end{document} 